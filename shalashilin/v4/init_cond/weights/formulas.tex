\documentclass[a4paper,14pt]{extarticle}
% Стандартные формульные пакеты
\usepackage{float,amsmath,esint,amsfonts,wrapfig,bbm}
\usepackage{indentfirst}
\usepackage[usenames]{color}
\usepackage{multirow}
%выставляем поля
\usepackage[left=2cm,right=2cm,top=2cm,bottom=2cm,bindingoffset=0cm]{geometry}
% Русский текст в формулах
\usepackage{mathtext}
% Подключение русского языка
\usepackage[T2A]{fontenc}
\usepackage[english,russian]{babel}
\usepackage[utf8]{inputenc}
% Рисунки
\usepackage{graphicx,caption,subcaption}
% Landscape page
\usepackage{lscape}
\renewcommand{\arraystretch}{1.1}

\begin{document}
Выберем базис гауссовых волновых пакетов, центрированных в фазовом пространстве в точках $(a,0),\ (-a,0),\ (0,a),\ (0,-a)$, где $a$ --- вариьируемый параметр.
Данные функции имеют вид:
$$\psi_1(x|a) = e^{ -0.5\omega(x - a)^2 }$$
$$\psi_2(x|a) = e^{ -0.5\omega(x + a)^2 }$$
$$\psi_3(x|a) = e^{ -0.5\omega x^2 + iax }$$
$$\psi_4(x|a) = e^{ -0.5\omega x^2 - iax }$$

Изучим, как выглядят вариации функций $\psi_i$. Для этого запишем выражения для $\psi_i(x|a+\delta a)$:
$$\psi_1(x|a+\delta a) = e^{ -0.5\omega(x - a - \delta a)^2 } = e^{ -0.5\omega(x - a)^2 + \omega(x-a)\delta a} = $$
$$ = e^{ -0.5\omega(x - a)^2} e^{\omega(x-a)\delta a} = \psi_1(1 + \omega(x-a)\delta a ) = \psi_1 + \omega(x-a)\psi_1\delta a$$

Получили, что первая вариация функции $\psi_1$ равна $\omega(x-a)\psi_1$. Аналогично, получим и значения первых вариаций остальных функций:
$$\delta\psi_2 = -\omega(x+a)\psi_2\delta a$$
$$\delta\psi_3 = ix\psi_3\delta a$$
$$\delta\psi_4 = -ix\psi_4\delta a$$

В общем виде:
$$\psi_i = e^{f_i},\ \delta\psi_i = \frac{d f_i}{da}\psi_i\delta a$$

Запишем стационарное уравнение Шредингера:
$$\hat{H}\tilde{\psi}_m = E_m\tilde{\psi}_m$$

Пусть собственные функции $\tilde{\psi}_m$ могут быть разложены по выбранному базису следующим образом:
$$ \tilde{\psi}_m = \sum_{i=1}^4 \psi_i D_{im}$$

Подставим разложение в стационарную задачу, умножим слева на $\psi_j^*$ и проинтегрируем. Получим:
$$\sum_{i=1}^4 D_{im} \langle\psi_j|\hat{H}|\psi_i\rangle = \sum_{i=1}^4 E_mD_{im}\langle\psi_j|\psi_i\rangle$$

Изучим поведение решения при замене параметра $a$ на $a + \delta a$ в базисных функциях:
$$\sum_{i=1}^4 D_{im} \langle\psi_j + \delta\psi_j|\hat{H}|\psi_i + \delta\psi_i\rangle = \sum_{i=1}^4 E_mD_{im}\langle\psi_j + \delta\psi_j|\psi_i + \delta\psi_i\rangle$$
$$\sum_{i=1}^4 D_{im} \left( \langle\psi_j|\hat{H}|\psi_i\rangle + \langle\delta\psi_j|\hat{H}|\psi_i\rangle + \langle\psi_j|\hat{H}|\delta\psi_i\rangle  \right) = $$
$$ = \sum_{i=1}^4 E_mD_{im}\left( \langle\psi_j|\psi_i\rangle + \langle\delta\psi_j|\psi_i\rangle + \langle\psi_j|\delta\psi_i\rangle \right)$$

$$\sum_{i=1}^4 D_{im} \left( \langle\psi_i|\hat{H}|\delta\psi_j\rangle^* + \langle\psi_j|\hat{H}|\delta\psi_i\rangle  \right) = %
  \sum_{i=1}^4 E_mD_{im}\left( \langle\psi_i|\delta\psi_j\rangle^* + \langle\psi_j|\delta\psi_i\rangle \right)$$

$$\sum_{i=1}^4 D_{im} \left( \langle\psi_i|\hat{H}\frac{d f_j}{da}|\psi_j\rangle^* + \langle\psi_j|\hat{H}\frac{d f_i}{da}|\psi_i\rangle  \right) = %
  \sum_{i=1}^4 E_mD_{im}\left( \langle\psi_i|\frac{d f_j}{da}|\psi_j\rangle^* + \langle\psi_j|\frac{d f_i}{da}|\psi_i\rangle \right)$$

$$\sum_{i=1}^4 D_{im} \left( \mathbb{A}_{ij}^* + \mathbb{A}_{ji}  \right) = %
  \sum_{i=1}^4 E_mD_{im}\left( \mathbb{B}_{ij}^* + \mathbb{B}_{ji} \right)$$

$$\sum_{i=1}^4 D_{im} \left( \mathbb{A}^{\dagger} + \mathbb{A} \right)_{ji} = %
  \sum_{i=1}^4 E_mD_{im}\left( \mathbb{B}^{\dagger} + \mathbb{B} \right)_{ji}$$

$$\left( \left( \mathbb{A}^{\dagger} + \mathbb{A} \right) \mathbb{D}  \right)_{jm} = E_m \left( \left( \mathbb{B}^{\dagger} + \mathbb{B} \right) \mathbb{D} \right)_{jm}$$
$$\left( \mathbb{A}^{\dagger} + \mathbb{A} \right) \mathbb{D} = \mathbb{E} \left( \mathbb{B}^{\dagger} + \mathbb{B} \right) \mathbb{D}$$
где $\mathbb{E}$ --- диагональная матрица.

\end{document}
\grid
