Итак, мы хотим:
\begin{enumerate}
\item Иметь "взвешенный" след $\Rightarrow$ "взвешенный" градиент
\item Веса должны иметь смысл вероятностей (сумма равна единице)
\item Градиент должен иметь "простой" вид и понятный физический смысл (желательно)
\end{enumerate}

В общем виде это можно записать следующим образом:
$$\mathit{eTr}(\mathbbm{H})=\sum_{i=1}^N\mathbbm{H}_{ii}P_i, \sum_{i=1}^NP_i=1$$

Рассмотрим такое выражение для вероятности:
$$P_i=\frac{1}{2}\sum_{j=1}^N\mathbbm{S}_{ij}\left(D_i^*D_j+D_iD_j^*\right)$$
$$\sum_{i=1}^NP_i=\left(\vec{D}^{\dagger}\mathbbm{S}\vec{D}\right)=1$$
$$\mathit{eTr}(\mathbbm{H})=\sum_{i=1}^N\mathbbm{H}_{ii}P_i,\ %
  \frac{d}{dt}\mathit{eTr}(\mathbbm{H})=\sum_{i=1}^N\dot{\mathbbm{H}}_{ii}P_i+%
						    \mathbbm{H}_{ii}\dot{P}_i$$
$$\dot{P}_i=\frac{1}{2}\sum_{j=1}^N\mathbbm{S}_{ij}\left(\dot{D}_i^*D_j+%
							 D_i^*\dot{D}_j+%
							 \dot{D}_iD_j^*+%
							 D_i\dot{D}_j^*%
						    \right)$$
$$\mathbbm{R}_{mn}=\mathbbm{H}_{mn}-i\hbar\tau_{mn}$$

$$\dot{D}_i^*D_j=\frac{i}{\hbar}\sum_{m,n=1}^N\left(\mathbbm{S}^{-1}\right)_{im}%
						    \mathbbm{R}_{mn}D_n^*D_j$$
$$D_i^*\dot{D}_j=-\frac{i}{\hbar}\sum_{m,n=1}^N\left(\mathbbm{S}^{-1}\right)_{jm}%
						     \mathbbm{R}_{mn}D_i^*D_n$$
$$\dot{D}_iD_j^*=-\frac{i}{\hbar}\sum_{m,n=1}^N\left(\mathbbm{S}^{-1}\right)_{im}%
						     \mathbbm{R}_{mn}D_nD_j^*$$
$$D_i\dot{D}_j^*=\frac{i}{\hbar}\sum_{m,n=1}^N\left(\mathbbm{S}^{-1}\right)_{jm}%
						    \mathbbm{R}_{mn}D_n^*D_i$$
$$\dot{P}_i=\frac{i}{2\hbar}\sum_{j,m,n=1}^N\mathbbm{S}_{ij}\mathbbm{R}_{mn}%
					    \left(\left(\mathbbm{S}^{-1}\right)_{im}(D_n^*D_j-D_nD_j^*)+%
					          \left(\mathbbm{S}^{-1}\right)_{jm}(D_n^*D_i-D_nD_i^*)\right)=$$
$$=-\frac{1}{\hbar}\left(\sum_{n,m=1}^N\delta_{im}\mathbbm{R}_{mn}\mathit{Im}(D_n^*D_i)+%
			 \sum_{j,n,m=1}^N\mathbbm{S}_{ij}\mathbbm{R}_{mn}%
					 \left(\mathbbm{S}^{-1}\right)_{im}\mathit{Im}(D_n^*D_j)\right)=$$
$$=-\frac{1}{\hbar}\left(\sum_{n=1}^N\mathbbm{R}_{im}\mathit{Im}(D_n^*D_i)+%
			 \sum_{j,n,m=1}^N\mathbbm{S}_{ij}\mathbbm{R}_{mn}%
					 \left(\mathbbm{S}^{-1}\right)_{im}\mathit{Im}(D_n^*D_j)\right)=...$$
$$\left[\mathbbm{R}_{mn}=%
  \mathbbm{H}_{mn}-i\hbar\tau_{mn}=\mathbbm{S}_{mn}\mathbbm{A}_{mn}+\mymean{g_m}{V}{g_n}\right]$$
\begin{align*}
...=-\frac{1}{\hbar}&\left(\sum_{n=1}^N(\mathbbm{S}_{in}\mathbbm{A}_{in}+\mymean{g_i}{V}{g_n})\mathit{Im}(D_n^*D_i)+\right.\\
+&\left.		   \sum_{j,m,n=1}^N\mathbbm{S}_{ij}\left(\mathbbm{S}^{-1}\right)_{im}%
					    (\mathbbm{S}_{mn}\mathbbm{A}_{mn}+\mymean{g_m}{V}{g_n})\mathit{Im}(D_n^*D_j)\right)
\end{align*}

