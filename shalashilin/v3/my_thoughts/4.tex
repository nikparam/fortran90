Начнем с рассмотрения базиса из одного волнового пакета.
Изучим вопрос сохранения энергии:
$$<V> = \left(\frac{m\omega}{\hbar\pi}\right)^{1/2}%
	\int \myexp{\frac{1}{\hbar}\left(-m\omega x^2 + %
					 \left(2\mathit{Re}(\xi)-%
					 \frac{\hbar}{l}\right)x + %
					 2\mathit{Re}(\eta)\right)}\,dx=$$
$$=\left(\frac{m\omega}{\hbar\pi}\right)^{1/2}%
   \int \myexp{\frac{1}{\hbar}\left(-m\omega x^2 + %
					 \left(2m\omega q-%
					 \frac{\hbar}{l}\right)x - %
					 m\omega q^2\right)}\,dx=$$
$$=\left(\frac{m\omega}{\hbar\pi}\right)^{1/2}%
   \myexp{-\frac{m\omega}{\hbar} q^2}\int\myexp{-\frac{m\omega}{\hbar}%
						\left( x^2 - 2\left(q - %
						    \frac{\hbar}{2ml\omega}%
						\right)x
						\right)}\,dx=$$
$$=\myexp{-\frac{m\omega}{\hbar}\left(q^2-%
				     \left(q-%
				     \frac{\hbar}{2ml\omega}%
				     \right)^2\right)}=$$
$$=\myexp{-\frac{q}{l}+\frac{\hbar}{4ml^2\omega}}$$
$$<T> = \frac{\hbar\omega}{2}+%
	\frac{|\xi|^2}{2m}-%
	\frac{1}{2}m\omega^2%
	\left(\frac{\hbar}{2m\omega}+%
	\left(\frac{\mathit{Re}(\xi)}{m\omega}%
	\right)^2\right)=$$
$$=\frac{\hbar\omega}{4}+%
   \frac{m^2\omega^2q^2+p^2}{2m}-%
   \frac{1}{2}m\omega^2q^2 = \frac{\hbar\omega}{4}+\frac{p^2}{2m}$$
$$<H>'_t=<V>'_t+<T>'_t=\frac{p\dot{p}}{m}-%
		       \frac{\dot{q}}{l}\myexp{-\frac{q}{l}+%
					       \frac{\hbar}{4ml^2\omega}}=$$
$$=\frac{p}{m}\left(\dot{p}-\frac{1}{l}\myexp{-\frac{q}{l}+%
					      \frac{\hbar}{4ml^2\omega}}%
	      \right)$$
Если принять, что $\dot{p} = -V'_x(q)$, то получим
$$<H>'_t=\frac{p}{lm}\myexp{-\frac{q}{l}}%
	 \left(1-\myexp{\frac{\hbar}{4ml^2\omega}}\right)$$
Если же принять, что $\dot{p}=-<V'_x>=<V>/l$, то получим $<H>'_t=0$

Базис из двух функций:
$$\mathbbm{V} = \left( \begin{matrix} %
		\myexp{-\frac{q_1}{l}+\frac{\hbar}{4ml^2\omega}} & %
		V_{12} \\ %
		V_{12}^* & %
		\myexp{-\frac{q_2}{l}+\frac{\hbar}{4ml^2\omega}} \\ %
		\end{matrix} \right)$$
$$\mathbbm{T} = \left( \begin{matrix} % 
		       \frac{\hbar\omega}{4}+\frac{p_1^2}{2m} & %
		       T_{12} \\ %
		       T_{12}^* & %
		       \frac{\hbar\omega}{4}+\frac{p_2^2}{2m} %
		       \end{matrix} \right)$$

Обобщая предыдущие рассуждения, предполагаем, 
что среднее значение энергии теперь сохраняться не будет. 
Однако, след марицы оператора Гамильтона все еще представляет собой сумму классических функций 
Гамильтона отдельных пакетов и поправки. 
Таким образом, след матрицы оператора Гамильтона будет сохраняться с течением времени.

