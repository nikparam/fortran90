\documentclass[a4paper,14pt]{extarticle}
% Стандартные формульные пакеты
\usepackage{float,amsmath,esint,amsfonts,wrapfig,bbm}
\usepackage{indentfirst}
\usepackage[usenames]{color}
\usepackage{multirow}
%выставляем поля
\usepackage[left=2cm,right=2cm,top=2cm,bottom=2cm,bindingoffset=0cm]{geometry}
% Русский текст в формулах
\usepackage{mathtext}
% Подключение русского языка
\usepackage[T2A]{fontenc}
\usepackage[english,russian]{babel}
\usepackage[utf8]{inputenc}
% Рисунки
\usepackage{graphicx,caption,subcaption}
% Landscape page
\usepackage{lscape}
\renewcommand{\arraystretch}{1.1}

\newcommand{\mymean}[3]{\left\langle #1 \left| #2 \right| #3 \right\rangle}
\newcommand{\myint}[2]{\langle #1 | #2 \rangle}
\newcommand{\myexp}[1]{\text{exp}\left( #1 \right)}
\newcommand{\myl}[0]{\mathit{L}}
\begin{document}
\section{Гармонический осциллятор}
\subsection{Усреднение импульса --- полное усреднение градиента}
$$V = \frac{1}{2}m\omega^2x^2,\ \dot{q}_k=\frac{p_k}{m},\ \dot{p}_k=-\left.\frac{\partial V}{\partial x}\right|_{q_k}=-m\omega^2q_k $$
$$g_k = N\myexp{\frac{1}{\hbar}\left(-\frac{1}{2}m\omega(x-q_k)^2+ip_k(x-q_k)\right)} = \myexp{\frac{1}{\hbar}\left(-\frac{1}{2}m\omega x^2 + \xi_kx + \eta_k\right)}$$
$$\xi_k = m\omega q_k + ip_k,\ \eta_k = \hbar\ln{N} - \frac{1}{2}m\omega q_k^2 - iq_kp_k$$
$$\dot{\xi}_k=m\omega\dot{q}_k+i\dot{p}_k=\omega p_k - im\omega^2q_k = -i\omega(m\omega q_k + ip_k)=-i\omega\xi_k$$
$$\dot{\eta}_k=-m\omega q_k\dot{q}_k-i(\dot{q}_kp_k+q_k\dot{p}_k)=-\dot{q}_k(m\omega q_k + ip_k)-iq_k\dot{p}_k=-\frac{p_k\xi_k}{m}+im\omega^2 q_k^2$$
$$\mathbbm{S}_{mk}=\myint{g_m}{g_k}=\int\myexp{\frac{1}{\hbar}\left(-m\omega x^2+(\xi_m^*+\xi_k)x+\eta_m^*+\eta_k\right)}\,dx=$$
$$=\int\myexp{\frac{1}{\hbar}\left(-m\omega\left[x-\frac{\xi_m^*+\xi_k}{2m\omega}\right]^2+\frac{(\xi_m^*+\xi_k)^2}{4m\omega}+%
				   \eta_m^*+\eta_k\right)}\,dx=$$
$$=\myexp{\frac{1}{\hbar}\left(\frac{(\xi_m^*+\xi_k)^2}{4m\omega}+\eta_m^*+\eta_k\right)}\int\myexp{-\frac{m\omega}{\hbar}y^2}\,dy=$$
$$=\myexp{\frac{1}{\hbar}\left(\frac{(\xi_m^*+\xi_k)^2}{4m\omega}+\eta_m^*+\eta_k\right)}\sqrt{\frac{\hbar\pi}{m\omega}}$$
$$\mathbbm{S}_{kk}= N^2 \sqrt{ \frac{\hbar\pi}{m\omega} }=1\ \Rightarrow\ N = \left(\frac{m\omega}{\hbar\pi}\right)^{1/4}$$
$$\mymean{g_m}{x}{g_k} = \int x\cdot\myexp{\frac{1}{\hbar}\left(-m\omega x^2+(\xi_m^*+\xi_k)x+\eta_m^*+\eta_k\right)}\,dx=$$
$$=\hbar\int\frac{\partial}{\partial(\xi_m^*+\xi_k)}\myexp{\frac{1}{\hbar}\left(-m\omega x^2 + (\xi_m^*+\xi_k)x+\eta_m^*+\eta_k\right)}\,dx$$
$$=\hbar\frac{\partial\mathbbm{S}_{mk}}{\partial(\xi_m^*+\xi_k)}=\frac{(\xi_m^*+\xi_k)}{2m\omega}\mathbbm{S}_{mk}$$
$$\mymean{g_m}{x^2}{g_k}=\int x^2\cdot\myexp{\frac{1}{\hbar}\left(-m\omega x^2+(\xi_m^*+\xi_k)x+\eta_m^*+\eta_k\right)}\,dx=$$
$$=\hbar^2\frac{\partial^2\mathbbm{S}_{mk}}{\partial(\xi_m^*+\xi_k)^2}=\frac{\hbar}{2m\omega}+\left(\frac{\xi_m^*+\xi_k}{2m\omega}\right)^2$$
$$\tau_{mk}=\myint{g_m}{\dot{g}_k}=\frac{1}{\hbar}\mathbbm{S}_{mk}\left( \dot{\eta}_k+\dot{\xi}_k\frac{\xi_k+\xi_m^*}{2m\omega}\right)=%
				   \frac{1}{\hbar}\mathbbm{S}_{mk}\left(im\omega^2 q_k^2-\frac{p_k\xi_k}{m}-i\frac{\xi_k^2+\xi_k\xi_m^*}{2m}\right)=$$
$$=\frac{1}{\hbar}\mathbbm{S}_{mk}\left(im\omega^2 q_k^2 - \frac{i\xi_k\xi_m^*}{2m} - \frac{i\xi_k(\xi_k-2ip_k)}{2m}\right)=%
   \frac{1}{\hbar}\mathbbm{S}_{mk}\left(im\omega^2 q_k^2 - \frac{i\xi_k\xi_m^*}{2m} - \frac{i|\xi_k|^2}{2m} \right)=$$
$$=\frac{1}{\hbar}\mathbbm{S}_{mk}\left(im\omega^2 q_k^2 - \frac{i\xi_k\xi_m^*}{2m} - \frac{i(m^2\omega^2q_k^2 + p_k^2)}{2m} \right)=%
   \frac{1}{\hbar}\mathbbm{S}_{mk}\left(\frac{i}{2}m\omega^2q_k^2 - \frac{ip_k^2}{2m} - \frac{i\xi_k\xi_m^*}{2m}\right)$$
$$\mathbbm{H}_{mk} = \mymean{g_m}{\hat{T}}{g_k} + \frac{1}{2}m\omega^2\mymean{g_m}{x^2}{g_k} =%
		     -\frac{\hbar^2}{2m}\mymean{g_m}{\frac{\partial^2}{\partial x^2}}{g_k} + \frac{1}{2}m\omega^2\mymean{g_m}{x^2}{g_k}=$$
$$=-\frac{\hbar^2}{2m}\mymean{g_m}{\frac{1}{\hbar}\frac{\partial}{\partial x} (-m\omega x + \xi_k)}{g_k} + \frac{1}{2}m\omega^2\mymean{g_m}{x^2}{g_k}=$$
$$=\frac{\hbar\omega}{2}\myint{g_m}{g_k} - \frac{1}{2m}\mymean{g_m}{(-m\omega x + \xi_k)^2}{g_k} + \frac{1}{2}m\omega^2\mymean{g_m}{x^2}{g_k}=$$
$$=\left(\frac{\hbar\omega}{2}-\frac{\xi_k^2}{2m}\right)\myint{g_m}{g_k}+\omega\xi_k\mymean{g_m}{x}{g_k}-%
	 \frac{1}{2}m\omega^2\mymean{g_m}{x^2}{g_k} + \frac{1}{2}m\omega^2\mymean{g_m}{x^2}{g_k}=$$
$$=\left(\left(\frac{\hbar\omega}{2}-\frac{\xi_k^2}{2m}\right)+\omega\xi_k\frac{(\xi_m^*+\xi_k)}{2m\omega}\right)\myint{g_m}{g_k}=%
   \left(\frac{\hbar\omega}{2}+\frac{\xi_m^*\xi_k}{2m}\right)\myint{g_m}{g_k}$$
$$\mathbbm{H}_{mk} - i\hbar\boldsymbol{\tau}_{mk} = \left( \frac{\hbar\omega}{2} + \frac{\xi_m^*\xi_k}{2m} + %
							   \frac{1}{2}m\omega^2q_k^2- \frac{p_k^2}{2m} -\frac{\xi_m^*\xi_k}{2m} \right)\myint{g_m}{g_k}=$$
$$=\left( \frac{\hbar\omega}{2} + \frac{1}{2}m\omega^2q_k^2 - \frac{p_k^2}{2m} \right)\mathbbm{S}_{mk}$$
$$( -i\mathbbm{S}^{-1}\left(\mathbbm{H}-i\hbar\boldsymbol{\tau}\right) )_{nk} = %
    -i\sum_m \mathbbm{S}^{-1}_{nm}\left( \mathbbm{H}_{mk} - i\hbar\boldsymbol{\tau}_{mk} \right) = $$
$$=-i\left( \frac{\hbar\omega}{2} +\frac{1}{2}m\omega^2q_k^2 - \frac{p_k^2}{2m} \right)\sum_m\mathbbm{S}^{-1}_{nm}\mathbbm{S}_{mk} = %
   -i\left( \frac{\hbar\omega}{2} +\frac{1}{2}m\omega^2q_k^2 - \frac{p_k^2}{2m} \right)\delta_{nk}$$
$$\dot{C}_n = -\frac{i}{\hbar}\sum_k \left(\frac{\hbar\omega}{2}+\frac{1}{2}m\omega^2q_k^2-\frac{p_k^2}{2m}\right)\delta_{nk}C_k=%
	      -\frac{i}{\hbar}\left(\frac{\hbar\omega}{2}+\frac{1}{2}m\omega^2q_n^2-\frac{p_n^2}{2m}\right)C_n$$
$$C_n(t) = C_n(0)\ \myexp{-\frac{i}{\hbar}\int_0^t\left(\frac{\hbar\omega}{2}+\frac{1}{2}m\omega^2q_n^2-\frac{p_n^2}{2m}\right)\,dt'}$$


Базис --- три волновых пакета: $\{q_k\}_{k=1}^3 = \{0, \pm 0.2\},\ \langle p\rangle = \sqrt{2}$
\begin{figure}[H]
\centering
\includegraphics[scale = 0.5]{../all/images/width_1.pdf}
\caption{Зависимость дисперсии кординаты ($\langle x^2\rangle-\langle x\rangle^2$) от времени}
\end{figure}
\begin{figure}[H]
\centering
\includegraphics[scale = 0.5]{../all/images/mean_q_t_1.pdf}
\caption{Зависимость математического ожидания координаты ($\langle x\rangle$) от времени}
\end{figure}
\begin{figure}[H]
\centering
\includegraphics[scale = 0.5]{../all/images/phase_1.pdf}
\caption{Фазовые профили базисных функций}
\end{figure}

\newpage

\subsection{Средний импульс --- локальное усреднение градиента}
Пускай все волновые пакеты движутся с одинаковым импульсом, изменение которого расчитывается в центре функции $|\Psi\rangle$
$$|g_k\rangle = N\myexp{\frac{1}{\hbar}\left(-\frac{1}{2}m\omega(x-q_k)^2+i\langle p\rangle x\right)} = %
		 \myexp{\frac{1}{\hbar}\left(-\frac{1}{2}m\omega x^2 + \xi_k x + \eta_k\right)}$$
$$\xi_k = m\omega q_k + i\langle p\rangle, \eta_k = \hbar\ln N - \frac{1}{2}m\omega q_k^2$$
$$\dot{\xi}_k = m\omega \dot{q}_k+i\langle\dot{p} \rangle = %
	         \omega\langle p\rangle - i\left\langle\frac{dV}{dx}\right\rangle$$
$$\left\langle\frac{dV}{dx}\right\rangle = %
  m\omega^2\langle x\rangle %
= \sum_{m,k}D_m^*D_k\mymean{g_m}{x}{g_k}=%
  \sum_{m,k}D_m^*D_k\mathbbm{S}_{mk}\frac{q_m+q_k}{2}$$
$$\dot{\eta}_k = -m\omega q_k\dot{q}_k = -\omega q_k\langle p\rangle$$
Матричные элементы примут вид:
\begin{enumerate}
\item матрица перекрывания:
$$\mathbbm{S}_{mk}=\sqrt{\frac{\hbar\pi}{m\omega}}\myexp{\frac{1}{\hbar}\left(\frac{(m\omega q_k+m\omega q_m)^2}{4m\omega}+%
									      2\hbar\ln N-\frac{1}{2}m\omega(q_k^2+q_m^2)\right)}=$$
$$=\myexp{\frac{1}{\hbar}\left(\frac{1}{4}m\omega(q_m+q_k)^2-\frac{1}{2}m\omega(q_m^2+q_k^2)\right)}=$$
$$=\myexp{-\frac{m\omega}{4\hbar}(q_m-q_k)^2}$$
\item матрица первых моментов:
$$\mymean{g_m}{x}{g_k}=\frac{q_m+q_k}{2}\mathbbm{S}_{mk}$$
\item матрица вторых моментов:
$$\mymean{g_m}{x^2}{g_k}=\left(\frac{\hbar}{2m\omega}+\left(\frac{q_m+q_k}{2}\right)^2\right)\mathbbm{S}_{mk}$$
\item матрица $\tau$:
$$\tau_{mk} = \myint{g_m}{\dot{g}_k}=\frac{1}{\hbar}\left(\dot{\eta}_k+\dot{\xi}_k\frac{(q_k+q_m)}{2}\right)\mathbbm{S}_{mk}=$$
$$=\frac{1}{\hbar}\left(-\omega q_k\langle p\rangle+\frac{1}{2}(\omega\langle p\rangle-im\omega^2\langle x\rangle)(q_m+q_k)\right)\mathbbm{S}_{mk}=$$
$$=\frac{1}{\hbar}\left(\frac{1}{2}\omega\langle p\rangle(q_m-q_k) - \frac{i}{2}m\omega^2\langle x\rangle(q_m+q_k)\right)\mathbbm{S}_{mk}$$
\item матрица оператора Гамильтона:
$$\mathbbm{H}_{mk} = \left(\frac{\hbar\omega}{2}+\frac{(m\omega q_m-i\langle p\rangle)(m\omega q_k+i\langle p\rangle)}{2m}\right)\mathbbm{S}_{mk}=$$
$$=\left(\frac{\hbar\omega}{2}+\frac{1}{2}m\omega^2 q_mq_k+\frac{\langle p\rangle^2}{2m}+\frac{i}{2}\omega\langle p\rangle(q_m-q_k)\right)\mathbbm{S}_{mk}$$
\end{enumerate}

Соберем матричные элементы:
$$\mathbbm{H}_{mk}-i\hbar\tau_{mk}=\left(\frac{\hbar\omega}{2}+\frac{1}{2}m\omega^2(q_mq_k-\langle x\rangle(q_m+q_k))+\frac{\langle p\rangle^2}{2m}\right)\mathbbm{S}_{mk}$$
Таким образом, матрица $\mathbbm{H}-i\hbar\tau$ действительна.
$$\dot{\vec{D}}=-\frac{i}{\hbar}\mathbbm{S}^{-1}(\mathbbm{H}-i\hbar\tau)\vec{D}$$
$$\myint{\Psi}{\Psi}=\vec{D}^{\dagger}\mathbbm{S}\vec{D},\ %
  \frac{d}{dt}\myint{\Psi}{\Psi}=2Re(\vec{D}^{\dagger}\mathbbm{S}\dot{\vec{D}})+\vec{D}^{\dagger}\dot{\mathbbm{S}}\vec{D}$$
$$\vec{D}^{\dagger}\mathbbm{S}\dot{\vec{D}}=%
  -\frac{i}{\hbar}\vec{D}^{\dagger}\mathbbm{S}\mathbbm{S}^{-1}(\mathbbm{H}-i\hbar\tau)\vec{D}=%
  -\frac{i}{\hbar}\vec{D}^{\dagger}(\mathbbm{H}-i\hbar\tau)\vec{D}$$
$$\dot{\vec{D}}^{\dagger}\mathbbm{S}\vec{D}+\vec{D}^{\dagger}\mathbbm{S}\dot{\vec{D}}=%
  \frac{i}{\hbar}\vec{D}^{\dagger}\left(\mathbbm{H}+i\hbar\tau^{\dagger}-\mathbbm{H}+i\hbar\tau\right)\vec{D}=%
  -\vec{D}^{\dagger}\left(\tau^{\dagger}+\tau\right)\vec{D}=-\vec{D}^{\dagger}\dot{\mathbbm{S}}\vec{D}$$
$$\dot{\mathbbm{S}}_{mk}= \myint{\dot{g}_m}{g_k}+\myint{g_m}{\dot{g}_k}=$$
$$=\frac{1}{\hbar}\mathbbm{S}_{mk}\left(\dot{\eta}_m^*+\dot{\eta}_k+\frac{1}{2}(\dot{\xi}_m^*+\dot{\xi}_k)(q_m+q_k)\right)=$$
$$=\frac{1}{\hbar}\mathbbm{S}_{mk}\left(-\omega\langle p\rangle(q_m+q_k)+\omega\langle p\rangle(q_m+q_k)\right)=0$$
Тогда $\frac{d}{dt}\myint{\Psi}{\Psi} = 0$



Базис --- три волновых пакета: $\{q_k\}_{k=1}^3 = \{0, \pm 0.2\},\ \langle p\rangle = \sqrt{2}$
\begin{figure}[H]
\centering
\includegraphics[scale = 0.5]{../all/images/width_2.pdf}
\caption{Зависимость дисперсии кординаты ($\langle x^2\rangle-\langle x\rangle^2$) от времени}
\end{figure}
\begin{figure}[H]
\centering
\includegraphics[scale = 0.5]{../all/images/mean_q_t_2.pdf}
\caption{Зависимость математического ожидания координаты ($\langle x\rangle$) от времени}
\end{figure}
\begin{figure}[H]
\centering
\includegraphics[scale = 0.5]{../all/images/phase_2.pdf}
\caption{Фазовые профили базисных функций}
\end{figure}

\end{document}
