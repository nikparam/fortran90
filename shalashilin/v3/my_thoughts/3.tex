В качестве третьей альтернативы попробуем использовать одинаковый импульс, %
изменение которого будем рассчитывать с помощью локально усредненного градиента:
при усреднении градиент пудет браться в центрах базисных функций:
$$|g_k\rangle=%
  N\myexp{\frac{1}{\hbar}\left(-\frac{1}{2}m\omega(x-q_k)^2+i\langle p\rangle x\right)}=%
  \myexp{\frac{1}{\hbar}\left(-\frac{1}{2}m\omega x^2 + \xi_k x + \eta_k\right)}$$
$$\xi_k = m\omega x + i\langle p\rangle,\ \eta_k = \hbar\ln N - \frac{1}{2}m\omega q_k^2$$
$$\dot{\xi}_k=m\omega\dot{q}_k+i\langle\dot{p}\rangle = %
              \omega\langle p\rangle - i \left\langle\left.\frac{dV}{dx}\right|_q\right\rangle=%
	      \omega\langle p\rangle - i\sum_{m,k}D_m^*D_k\mymean{g_m}{\left.\frac{dV}{dx}\right|_{q_k}}{g_k}=$$
$$=\omega\langle p\rangle - im\omega^2\sum_{m.k}D_m^*D_k\mathbbm{S}_{mk}q_k = %
   \omega\langle p\rangle - im\omega^2 \{x\}$$
$$\dot{\eta}_k = -m\omega q_k\dot{q}_k = -\omega q_k\langle p\rangle$$

При этом элементы матрицы Гамильтониана не изменятся:
$$\mathbbm{H}_{mk} = \left(\frac{\hbar\omega}{2} + %
			   \frac{1}{2}m\omega^2q_mq_k + %
			   \frac{\langle p\rangle^2}{2m} + %
			   \frac{i}{2}\omega\langle p\rangle(q_m-q_k)\right)\mathbbm{S}_{mk}$$

А в матрице $\tau$ изменится только форма расчета среднего значения координаты:
$$\tau_{mk}=\myint{g_m}{\dot{g}_k} = %
	    \frac{1}{\hbar}\mathbbm{S}_{mk}\left(\dot{\eta}_k+%
						 \dot{\xi}_k\frac{q_k+q_m}{2}\right) = $$
$$=\frac{1}{\hbar}\mathbbm{S}_{mk}\left(-\omega q_k\langle p\rangle + % 
                          (\omega\langle p\rangle-im\omega^2\{x\})\frac{q_k+q_m}{2}\right) = $$
$$=\frac{1}{\hbar}\mathbbm{S}_{mk}\left(\omega\langle p\rangle\frac{q_m-q_k}{2} - %
			 im\omega^2\{x\}\frac{q_m+q_k}{2}\right)$$

При объединении видим, что выражении по форме совпадает с таковым, полученным при полном усреднении:
$$\mathbbm{H}_{mk}-i\hbar\tau_{mk} = %
  \left(\frac{\hbar\omega}{2} + %
	\frac{1}{2}m\omega^2(q_mq_k-\{x\}(q_m+q_k)) + %
	\frac{\langle p\rangle^2}{2m}\right)$$
Снова видим, что $\mathbbm{H}-i\hbar\tau$ действительна. 
$$\dot{\vec{D}} = -\frac{i}{\hbar}\mathbbm{S}^{-1}\left(\mathbbm{H}-i\hbar\tau\right)\vec{D}$$
$$\myint{\Psi}{\Psi} = \vec{D}^{\dagger}\mathbbm{S}\vec{D},\ %
  \frac{d}{dt}\myint{\Psi}{\Psi} = 2Re(\vec{D}^{\dagger}\mathbbm{S}\dot{\vec{D}})+%
				   \vec{D}^{\dagger}\dot{\mathbbm{S}}\vec{D}$$
Первое слагаемое снова равно $0$, изучим $\dot{\mathbbm{S}}$:
$$\dot{\mathbbm{S}}_{mk} = \myint{\dot{g}_m}{g_k}+\myint{g_m}{\dot{g}_k} = %
  \frac{1}{\hbar}\mathbbm{S}_{mk}\left(\dot{\eta}_m^*+\dot{\eta}_k+%
					(\dot{\xi}_m^*+\dot{\xi}_k)\frac{q_m+q_k}{2}\right)=$$
$$= \frac{1}{\hbar}\mathbbm{S}_{mk}\left(-\omega\langle p\rangle(q_m+q_k) + %
					  \omega\langle p\rangle(q_m+q_k)\right) = 0$$
Таким образом, данная динамика сохраняет норму.
